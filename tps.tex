The thin plate spline approach~\cite{Wahba1990} finds a function $f$ mapping between pairs of correspondence points that minimizes the smoothness of $f$:
\begin{equation}
\begin{align}
 &\underset{f}{\text{minimize\ \ \ }} & ||f||_{tps}^2 \\
 &\text{subject to\ \ \ }
 & \xt{i} = f(\xs{i}) \\
\end{align}
\label{eq:tps}
\end{equation}
where \xs{i} and \xt{i}, $i = 1,...,k$, refer to pairs of correspondence points in the source and target scenes, respectively. The objective $||f||_{tps}^2$ is the TPS energy function,
\begin{equation}
||f||_{tps}^2 = \int d\mathbf{x} || D^2 f(\mathbf{x})||_{frob}^2,
\label{eq:tps_energy}
\end{equation}
which is a measure of the curvature of the function.

Alternatively, we can trade off between matching correspondence points and the smoothness of our curve with the following regularized objective
\begin{equation}
f = \underset{f}{\text{argmin}} \sum_{i=1}^k ||\xt{i} - f(\xs{i})||_2^2 + \lambda ||f||_{tps}^2,
\label{eq:tps_reg}
\end{equation}
where the parameter $\lambda$ controls this tradeoff.

Duchon showed the minimizer $f$ in Equation~\ref{eq:tps} can be expressed as a weighted sum of basis functions centered around the data points \xs{i}, in addition to an affine part~\cite{Duchon1977}. First, we define the basis functions $U$ in dimension $d$ as
\begin{equation}
U(\mathbf{x}) = \begin{cases}
    c_0 ||\mathbf{x}||^{4-d} \text{ ln } ||\mathbf{x}||, & d = 2 \text{ or } d = 4\\
    c_1 ||\mathbf{x}||^{4-d}, & \text{otherwise},
  \end{cases}
\end{equation}
where $c_0$ and $c_1$ are constants.
After factoring in the constant factor, when $d = 2$ we have $U(\mathbf{x}) = ||\mathbf{x}||^2 \ln||\mathbf{x}||$, and when $d = 3$ we have $U(\mathbf{x}) = -||\mathbf{x}||$.

Then $f$ can be expressed as
\begin{equation}
f(\mathbf{x}) = 
\begin{bmatrix}
A \\
B \\
\mathbf{c}
\end{bmatrix}^\top
\mathbf{b}
\end{equation}
where $A$, $B$ and $\mathbf{c}$ are some weight matrices and $\mathbf{b}$ is a vector of the basis functions
\begin{equation}
\mathbf{b} =
\begin{bmatrix} 
  U(\mathbf{x} - \xs{1}) & U(\mathbf{x} - \xs{2}) & \cdots & U(\mathbf{x} - \xs{k}) & \mathbf{x}^\top & 1
\end{bmatrix}^\top .
\end{equation}

The integral in Equation~\ref{eq:tps_energy} is only finite if the weights $A_{ij}$ of the non-affine part have vanishing moments
\begin{equation}
\sum_{i=1}^k A_{ij} \xs{i} = \mathbf{0}
\text{\ and \ }
\sum_{i=1}^k A_{ij} = 0
\text{\ \ }
\forall j=1,...,d
.
\end{equation}

Given this constraint, the TPS energy function can be expressed in closed-form:
\begin{equation}
||f||_{tps}^2 = \Tr(A^\top K A)
\end{equation}
where $K$ is the kernel matrix
\begin{equation}
K = 
\begin{bmatrix} 
  U(\xs{1} - \xs{1}) & \cdots & U(\xs{1} - \xs{k}) \\
  U(\xs{2} - \xs{1}) & \cdots & U(\xs{2} - \xs{k}) \\
  \vdots &\ddots & \vdots \\
  U(\xs{k} - \xs{1}) & \cdots & U(\xs{k} - \xs{k})
\end{bmatrix}
.
\end{equation}

Thus, the optimization problem of Equation~\ref{eq:tps} has a closed form solution as well; the weight matrices are given by solving the linear system
\begin{equation}
\begin{bmatrix} 
  K & Q\\
  Q^\top & \mathbf{0} 
\end{bmatrix}
\begin{bmatrix}
A \\
B \\
\mathbf{c}
\end{bmatrix} = D
\end{equation}

where

\begin{equation}
Q = 
\begin{bmatrix} 
  \xs{1}^\top & 1 \\
  \xs{2}^\top & 1 \\
  \vdots & \vdots \\
  \xs{k}^\top & 1
\end{bmatrix}
\text{ and }
D = 
\begin{bmatrix} 
  \xt{1}^\top \\
  \xt{2}^\top \\
  \vdots \\
  \xt{k}^\top \\
  \mathbf{0}
\end{bmatrix}
.
\end{equation*}
